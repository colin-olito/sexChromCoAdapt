% Preamble
\documentclass{article}

%Dependencies
\usepackage[left]{lineno}
%\usepackage{indentfirst}
\usepackage{titlesec}
\usepackage[utf8]{inputenc}
\usepackage{amsmath}
\usepackage{amsfonts}
\usepackage{amssymb}
%\usepackage{times}
\usepackage[sc]{mathpazo} %Like Palatino with extensive math support
\linespread{1.25}
% Default margins are too wide all the way around. I reset them here
\usepackage{fullpage}
% Force figures to be in correct section
\usepackage{placeins}

% highlighting command
\usepackage{xcolor}
\newcommand\hl[1]{%
  \bgroup
  \hskip0pt\color{blue!80!black}%
  #1%
  \egroup
}


% Fonts and language
\RequirePackage[utf8]{inputenc}
\RequirePackage[english]{babel}
\RequirePackage{amsmath,amsfonts,amssymb}
\RequirePackage{mathpazo}
\RequirePackage[scaled]{helvet}
\RequirePackage[T1]{fontenc}
\RequirePackage{url}
\RequirePackage[colorlinks=true, allcolors=blue]{hyperref}
%\RequirePackage[colorlinks=true, allcolors=blue,draft=true]{hyperref}
\RequirePackage{lettrine}
\RequirePackage{cleveref}

% Graphics, tables and other formatting
\RequirePackage{graphicx,xcolor}
\RequirePackage{colortbl}
\RequirePackage{booktabs}
\RequirePackage{tikz}
\RequirePackage{algorithm}
\RequirePackage[noend]{algpseudocode}
\RequirePackage{changepage}
\RequirePackage[labelfont={bf,sf},%
                labelsep=space,%
                figurename=Figure,%
                singlelinecheck=off,%
                justification=RaggedRight]{caption}
%\setlength{\columnsep}{24pt} % Distance between the two columns of text
\setlength{\parindent}{12pt} % Paragraph indent

% Bibliography
\usepackage{natbib} \bibpunct{(}{)}{;}{author-year}{}{,}
\bibliographystyle{amnatnat}
\addto{\captionsenglish}{\renewcommand{\refname}{Literature Cited}}
\setlength{\bibsep}{0.0pt}

% Running headers
\usepackage{fancyhdr}
\setlength{\headheight}{28pt}

% Graphics package
\usepackage{graphicx}
\graphicspath{{../../output/figures/}.pdf}

% New commands: fonts
%\newcommand{\code}{\fontfamily{pcr}\selectfont}
%\newcommand*\chem[1]{\ensuremath{\mathrm{#1}}}
\newcommand\numberthis{\addtocounter{equation}{1}\tag{\theequation}}
\def\mathbi#1{\textbf{\em #1}}

%% Other Options

% Change subsection numbering
\renewcommand\thesubsection{\arabic{subsection})}
\renewcommand\thesubsubsection{}

% Turn off section numbering without disrupting hyperref
\makeatletter
\renewcommand\@seccntformat[1]{}
\makeatother

% Subsubsection Title Formatting
\titleformat{\subsubsection}    
{\normalfont\fontsize{12pt}{17}\itshape}{\thesubsubsection}{12pt}{}


%%%%%%%%%%%%%%%%%%%%%%%%%%%%%%%%%%%%%%%%%%%%
\title{Supplementary Material (Appendices A--X) for: Sexually antagonistic coevolution between the sex chromosomes. \hl{\textit{PNAS}}}

\author{Katrine K.~Lund-Hansen, Colin Olito, Edward H.~Morrow, \& Jessica K.~Abbott}
\date{\today}

\begin{document}
\maketitle

\noindent{} $^{2}$ Department of Biology, Section for Evolutionary Ecology, Lund University, Lund 223 62, Sweden.

\noindent{} $^{\ast}$ Corresponding author e-mail: \url{katrine.lund-hansen@biol.lu.se}

\newpage
%%%%%%%%%%%%%%%%%%%%%%%%%%%%%%%%%%%%%%%%%%%%
% Running Header
\pagestyle{fancyplain}
\makeatother
%\lhead{\textit{Supplement to Olito \textit{et al.}. Coadaptation between the sex chromosomes. \textit{Targ.~Journ.}.\\}}
\lhead{\textit{Supplement to Lund-Hansen \textit{et al}. (2019). PNAS}\\ }
\rhead{\textit{Coadaptation between the sex chromosomes}\\ }
\renewcommand{\headrulewidth}{0pt}
\renewcommand{\footrulewidth}{0pt}
\addtolength{\headheight}{12pt}


\section{Appendix A: Development of the recursions}
\renewcommand{\theequation}{A\arabic{equation}}
\titleformat{\subsubsection}    
{\normalfont\fontsize{12pt}{17}\itshape}{\thesubsubsection}{12pt}{}

%\setcounter{subsection}{0}  % reset counter 
%\setcounter{equation}{0}  % reset counter 


Our models differ from standard selection models in two important ways: (i) mating is non-random (influenced by males' genotype at $\mathbf{Y}$, and (ii) the relative fitness of offspring depends on the parental genotypes rather than their own (i.e., selection in the current generation depends on genotypic frequencies among parents in the previous generation). Here, we develop the recursion equations underlying the deterministic version of the Autosomal and X-linked models described in the main text, as well as the eigenvalues for the relevant boundary equilibria used in subsequent analyses. For the deterministic models below, we assume sufficiently large population sizes that drift can be ignored, in addition to the major assumptions outlined in the main text. 

%%%%%%%%%%%%%%%%%%%%%%%%%%%%%%%%
\subsection{The Autosomal Model}

When the compensatory locus is autosomal, we can model the dynamics of the two interacting loci by tracking the frequency of mutant $y$ genotype, $q_{y}$, and each of the three autosomal genotypes at the compensatory locus ($F_{AA}$, $F_{Aa}$, $F_{aa}$). We begin with the recursion for the Y-linked locus, $\mathbf{Y}$.

Let $(1 - q_y)$ and $q_{y}$ equal the frequency of the wild-type ($Y$) and mutant ($y$) genotype among adult males prior to mating, with $w^{m}_{Y}$ and $w^{m}_{y}$ denoting the relative mating success of the two genotypes respectively. The genotypic frequencies among male gametes at mating are then 

\begin{subequations}\begin{align} \label{eq:Auto-qg}
	(1 - q^{m}_{y}) &= \frac{ (1 - q_y)w^{m}_{Y} }{\overline{w}_{m}} \\
	q^{m}_{y}       &= \frac{ q_y w^{m}_{y} }{\overline{w}_{m}}, 
\end{align}\end{subequations}

\noindent where $\overline{w}_{m} = (1 - q_y)w^{m}_{Y} + q_y w^{m}_{y}$ is the average fitness of both genotypes with respect to mating success. The resulting genotypic frequencies among offspring after fertilization and viability selection depends on the parent genotypes. We now write the frequency of the three possible genotypes at the autosomal compensatory locus as $F_{AA}$, $F_{Aa}$, and $F_{aa}$. Let $w^{o}_{Y:AA}$, $w^{o}_{Y:Aa}$, and $w^{o}_{Y:aa}$ denote the relative fitness of offspring sired by a wild-type male with each of the three possible female genotypes, with $w^{o}_{y:AA}$, $w^{o}_{y:Aa}$, and $w^{o}_{y:aa}$ representing the same for matings involving mutant males. The genotypic frequencies among offspring after fertilization and selection are then

\begin{subequations} \begin{align} 
	(1 - q'_{y}) &= \frac{ (1 - q^{m}_{y}) (F_{AA} w^{o}_{Y:AA} + F_{Aa} w^{o}_{Y:Aa} + F_{aa} w^{o}_{Y:aa}) }{\overline{w}^{\mathbf{Y}}_{o}} \label{eq:Auto-qPr1} \\ 
	q'_{y} &= \frac{ q^{m}_{y} (F_{AA} w^{o}_{y:AA} + F_{Aa} w^{o}_{Y:Aa} + F_{aa} w^{o}_{Y:aa}) }{\overline{w}^{\mathbf{Y}}_{o}}, \label{eq:Auto-qPr2} 
\end{align}\end{subequations}

\noindent where $\overline{w}^{\mathbf{Y}}_{o}$ represents mean offspring fitness, and is equal to the sum of the numerators of Eq(\ref{eq:Auto-qPr1}, \ref{eq:Auto-qPr2}).

Recalling that $q^{m}_{y}$ denotes the frequency of the mutant $y$ genotype among male gametes, the corresponding recursions for the compensatory locus, $\mathbf{A}$, are

\begin{subequations}\label{eq:Auto-ALocusRecursions} 
	\begin{align} 
		F'_{AA} &= \dfrac{ (1 - q^{m}_{y}) \bigg(F_{AA} w^{o}_{Y:AA} + \dfrac{F_{Aa}}{2} w^{o}_{Y:Aa} \bigg) (1 - q_a) + q^{m}_{y} \bigg(F_{AA} w^{o}_{y:AA} + \dfrac{F_{Aa}}{2} w^{o}_{y:Aa} \bigg) (1 - q_a)}{\overline{w}^{\mathbf{A}}_o} \label{eq:Auto-ARec1} \\ 
		F'_{Aa} &= \Bigg( (1 - q^{m}_{y}) \Bigg[ \bigg( F_{AA} w^{o}_{Y:AA} + \dfrac{F_{Aa}}{2} w^{o}_{Y:Aa} \bigg) q_a + \bigg( F_{aa} w^{o}_{Y:aa} + \dfrac{F_{Aa}}{2} w^{o}_{Y:Aa} \bigg) (1 - q_a) \bigg] +\nonumber \\
		&~~~~~~~~~~~~~~~q^{m}_{y} \Bigg[ \bigg( F_{AA} w^{o}_{y:AA} + \dfrac{F_{Aa}}{2} w^{o}_{y:Aa} \bigg) q_a	+ \bigg(F_{aa} w^{o}_{y:aa} + \dfrac{F_{Aa}}{2} w^{o}_{y:Aa} \bigg) (1 - q_a) \Bigg] \Bigg) \Bigg/ {\overline{w}^{\mathbf{A}}_o} \label{eq:Auto-ARec2}\\ 
		F'_{aa} &= \dfrac{ (1 - q^{m}_{y}) \bigg( F_{aa} w^{o}_{Y:aa} + \dfrac{F_{Aa}}{2} w^{o}_{Y:Aa} \bigg) q_a + q^{m}_{y} \bigg(F_{aa} w^{o}_{y:aa} + \dfrac{F_{Aa}}{2} w^{o}_{y:Aa} \bigg) q_a}{\overline{w}^{\mathbf{A}}_o}, \label{eq:Auto-ARec3}
	\end{align}
\end{subequations}

\noindent where $q_a = F_{aa} + F_{Aa}/2$, and $\overline{w}^{\mathbf{A}}_o$ is the sum of the numerators of Eqs(\ref{eq:Auto-ALocusRecursions}a-c).


%%%%%%%%%%%%%%%%%%%%%%%%%%%%
\subsubsection{Invasion of a mutant y chromosome}

During the initial stage of a coevolutionary cycle (invasion and subsequent increase in the frequency of $y$ to fixation in a population fixed for the wild-type compensatory allele), the evolutionary dynamics of the $\mathbf{Y}$ locus are independent of the genomic location of the compensatory locus. Substituting the fitness expressions defined in \hl{Table $1$} of the main text, the recursion equation for the expected frequency of the mutant $y$ allele in the next generation for a population initially fixed for the wild-type compensatory allele ($q_a = 0)$ reduces to:

\begin{equation} \label{eq:qPry}
	q'_{y} = \frac{q_y (1 + s_m)(1 - s_o)}{1 + q_y (s_m (1 - s_o) - s_o)}.
\end{equation}

\noindent From Eq(\ref{eq:qPry}), we see that the mutant $y$ chromosome can invade when $s_m - s_o(1 + s_m) > 0$. Under weak selection (dropping terms in $s_m s_o$), we see clearly that invasion of $y$ requires that $\delta = (s_m - s_o) > 0$, the increase in male mating success is greater than the loss of offspring viability relative to wild-type males.


%%%%%%%%%%%%%%%%%%%%%%%%%%%
\subsubsection{Invasion of autosomal compensatory mutation}

To analyze the linear stability of the system of recursion equations, Eqs(\ref{eq:Auto-qPr2}, \ref{eq:Auto-ALocusRecursions}a-c) at specific equilibria, we first define the Jacobian matrix:

\begin{equation}
	\mathbb{J}_{\text{A}} = \left( \begin{array}{cccc} 
		\dfrac{\partial  q'_y}{\partial q_y} & \dfrac{\partial  q'_y}{\partial F_{AA}} & \dfrac{\partial  q'_y}{\partial F_{Aa}} & \dfrac{\partial  q'_y}{\partial F_{aa}} \\ 
		\dfrac{\partial  F'_{AA}}{\partial q_y} & \dfrac{\partial  F'_{AA}}{\partial F_{AA}} & \dfrac{\partial  F'_{AA}}{\partial F_{Aa}} & \dfrac{\partial  F'_{AA}}{\partial F_{aa}} \\ 
		\dfrac{\partial  F'_{Aa}}{\partial q_y} & \dfrac{\partial  F'_{Aa}}{\partial F_{AA}} & \dfrac{\partial  F'_{Aa}}{\partial F_{Aa}} & \dfrac{\partial  F'_{Aa}}{\partial F_{aa}} \\ 
		\dfrac{\partial  F'_{aa}}{\partial q_y} & \dfrac{\partial  F'_{aa}}{\partial F_{AA}} & \dfrac{\partial  F'_{aa}}{\partial F_{Aa}} & \dfrac{\partial  F'_{aa}}{\partial F_{aa}} \\ 
	\end{array} \right)
\end{equation}

\noindent If a mutant $y$ chromosome sweeps to fixation prior to the occurrance of a compensatory mutation, completion of coevolutionary cycle requires that a mutant $a$ allele invades a population initially fixed for $y$ ($q_y = 1$). To determine the conditions for invasion of $a$ into a population initially fixed for $y$, we evaluated the candidate leading eigenvalue associated with invasion at $\mathbf{A}$, $\lambda_{\mathbf{A}}$ at the boundary equilibrium where $q_y = 1$ and $q_a = 0$.

\begin{equation} \label{eq:Lambda-a-qy1}
	\lambda_{\mathbf{A}|\hat{q}_y = 1,\hat{q}_a = 0} = \frac{(2 - s_o - h_o s_o)} {(2 (1 - s_o))}.
\end{equation}

\noindent Solving $\lambda_{\mathbf{A}|\hat{q}_y = 1,\hat{q}_a = 0} - 1 > 0$ yields the intuitive conclusion that invasion of the compensatory allele requires only that there is selection against the wild-type allele (i.e.,  $h_o < 1$ and $0 < s_o < 1$).

On the other hand, if a compensatory mutation arises before the mutant $y$ chromosome has fixed in the population, we must determine the conditions for invasion of a mutant $a$ allele into a population where $q_y$ is unspecified. Given these initial frequencies, and the fitness expressions provided in Table 1,

\begin{equation} \label{eq:Lambda-a}
	\lambda_{\mathbf{A}|\hat{q}_a = 0} = \frac{2(q_y s_m + 1) - h_c s_c(1 - q_y) - q_y s_o (1 + h_o)(1 + s_m)} {2 q_y (s_m (1 - s_o) - s_o) + 2}.
\end{equation}

\noindent Solving $\lambda_{\mathbf{A}|\hat{q}_a = 0} - 1 > 0$ for $q_y$ yields the threshold frequency of the mutant $y$ allele at which selection begins to favour the invasion of the $a$ allele

\begin{equation} \label{eq:aInvDelta-threshold}
	\tilde{q}_y^{\text{A}} = \frac{h_c s_c} {h_c s_c + (s_m - \delta)(1 - h_o)(1+s_m)}.
\end{equation}




%%%%%%%%%%%%%%%%%%%%%%%%%%%%%%%%
\subsection{The X-linked Model}

When the compensatory locus is located on the X chromosome, it is necessary to track the genotypic frequencies of all seven possible pairings of the sex chromosomes: $XY$, $Xy$, $xY$, and $xy$ for males, and $XX$, $Xx$, and $xx$ for females. The frequencies of the relevant chromosomes sex chromosomes among male gamete contributed by each male genotype are

\begin{subequations}\begin{align} \label{eq:Fg-Xlinked}
	F_{XY}^{g} &= \frac{F_{XY}w^{m}_{Y}}{\overline{w}_{m}} \\
	F_{Xy}^{g} &= \frac{F_{Xy}w^{m}_{Y}}{\overline{w}_{m}} \\
	F_{xY}^{g} &= \frac{F_{xY}w^{m}_{Y}}{\overline{w}_{m}} \\
	F_{xy}^{g} &= \frac{F_{xy}w^{m}_{Y}}{\overline{w}_{m}}, 
\end{align}\end{subequations}

\noindent where, $\overline{w}_{m}$ is now the sum of the numerators of Eqs(\ref{eq:Fg-Xlinked}). For simplicity, we also define the overall frequency of the $Y$ and $y$ chromosomes among male gametes as 

 \begin{subequations}\begin{align} \label{eq:qg-Xlinked}
	F_{\cdot Y}^{g} &= \frac{(F_{XY} + F_{xY})w^{m}_{Y}}{\overline{w}_{m}} \\
	F_{\cdot y}^{g} &= \frac{(F_{Xy} + F_{xy})w^{m}_{y}}{\overline{w}_{m}}. 
\end{align}\end{subequations}

The genotypic frequencies among male offspring after fertilization and selection can now be described by the following system of recursions

\begin{subequations}\label{eq:X-MaleRecursions} 
	\begin{align} 
		F'_{XY} &= \dfrac{ F_{\cdot Y}^g \bigg(F_{XX} w^{o}_{Y:XX} + \dfrac{F_{Xx}}{2} w^{o}_{Y:Xx} \bigg)}{\overline{w}^{\mathbf{X}}_{m}} \label{eq:X-XYRec1} \\ 
		F'_{Xy} &= \dfrac{ F_{\cdot y}^g \bigg(F_{XX} w^{o}_{y:XX} + \dfrac{F_{Xx}}{2} w^{o}_{y:Xx} \bigg)}{\overline{w}^{\mathbf{X}}_{m}} \label{eq:X-XyRec2} \\ 
		F'_{xY} &= \dfrac{ F_{\cdot Y}^g \bigg(F_{xx} w^{o}_{Y:xx} + \dfrac{F_{Xx}}{2} w^{o}_{Y:Xx} \bigg)}{\overline{w}^{\mathbf{X}}_{m}} \label{eq:X-xYRec3} \\ 
		F'_{xy} &= \dfrac{ F_{\cdot y}^g \bigg(F_{xx} w^{o}_{y:xx} + \dfrac{F_{Xx}}{2} w^{o}_{y:Xx} \bigg)}{\overline{w}^{\mathbf{X}}_{m}}, \label{eq:X-xyRec4}
	\end{align}
\end{subequations}

\noindent where $\overline{w}^{\mathbf{X}}_{m}$ is the sum of the numerators of Eqs(\ref{eq:X-MaleRecursions}a-d). The genotypic frequencies among female offspring after fertilization and selection are then

\begin{subequations}\label{eq:X-FemaleRecursions} 
	\begin{align} 
		F'_{XX} &= \dfrac{ F_{XY}^g \bigg(F_{XX} w^{o}_{Y:XX} + \dfrac{F_{Xx}}{2} w^{o}_{Y:Xx} \bigg) + F_{Xy}^g \bigg(F_{XX} w^{o}_{y:XX} + \dfrac{F_{Xx}}{2} w^{o}_{y:Xx} \bigg)}{\overline{w}^{\mathbf{X}}_{f}} \label{eq:X-XxRec1} \\ 
		F'_{Xx} &= \dfrac{1}{\overline{w}^{\mathbf{X}}_{f}} \Bigg[F_{XY}^g \bigg(F_{xx} w^{o}_{Y:xx} + \dfrac{F_{Xx}}{2} w^{o}_{Y:Xx} \bigg) + F_{xY}^g \bigg(F_{XX} w^{o}_{Y:XX} + \dfrac{F_{Xx}}{2} w^{o}_{Y:Xx} \bigg) + \nonumber\\
		&~~~~~~~~~~~~F_{Xy}^g \bigg(F_{xx} w^{o}_{y:XX} + \dfrac{F_{Xx}}{2} w^{o}_{y:Xx} \bigg) + F_{xy}^g \bigg(F_{XX} w^{o}_{y:XX} + \dfrac{F_{Xx}}{2} w^{o}_{y:Xx} \bigg) \Bigg]  \label{eq:X-XxRec2} \\ 
		F'_{xx} &= \dfrac{ F_{xY}^g \bigg(F_{xx} w^{o}_{Y:xx} + \dfrac{F_{Xx}}{2} w^{o}_{Y:Xx} \bigg) + F_{xy}^g \bigg(F_{xx} w^{o}_{y:xx} + \dfrac{F_{Xx}}{2} w^{o}_{y:Xx} \bigg)}{\overline{w}^{\mathbf{X}}_{f}}, \label{eq:X-xxRec3}
	\end{align}
\end{subequations}


%%%%%%%%%%%%%%%%%%%%%%%%%%%
\subsubsection{Invasion of mutant y an X-linked compensatory mutation}

To analyze the linear stability of the system of recursion equations, Eqs(\ref{eq:X-MaleRecursions}a-d, \ref{eq:X-FemaleRecursions}a-c), at specific equilibria, we first define the Jacobian matrix:

\begin{equation}
	\mathbb{J}_{\text{X}} = \left( \begin{array}{ccccccc} 
		\dfrac{\partial  F'_{XY}}{\partial F_{XY}} & \dfrac{\partial  F'_{XY}}{\partial F_{Xy}} & \dfrac{\partial  F'_{XY}}{\partial F_{xY}} & \dfrac{\partial  F'_{XY}}{\partial F_{xy}} & \dfrac{\partial  F'_{XY}}{\partial F_{XX}} & \dfrac{\partial  F'_{XY}}{\partial F_{Xx}} & \dfrac{\partial  F'_{XY}}{\partial F_{xx}} \\ 
		\dfrac{\partial  F'_{Xy}}{\partial F_{XY}} & \dfrac{\partial  F'_{Xy}}{\partial F_{Xy}} & \dfrac{\partial  F'_{Xy}}{\partial F_{xY}} & \dfrac{\partial  F'_{Xy}}{\partial F_{xy}} & \dfrac{\partial  F'_{Xy}}{\partial F_{XX}} & \dfrac{\partial  F'_{Xy}}{\partial F_{Xx}} & \dfrac{\partial  F'_{Xy}}{\partial F_{xx}} \\ 
		\dfrac{\partial  F'_{xY}}{\partial F_{XY}} & \dfrac{\partial  F'_{xY}}{\partial F_{Xy}} & \dfrac{\partial  F'_{xY}}{\partial F_{xY}} & \dfrac{\partial  F'_{xY}}{\partial F_{xy}} & \dfrac{\partial  F'_{xY}}{\partial F_{XX}} & \dfrac{\partial  F'_{xY}}{\partial F_{Xx}} & \dfrac{\partial  F'_{xY}}{\partial F_{xx}} \\ 
		\dfrac{\partial  F'_{xy}}{\partial F_{XY}} & \dfrac{\partial  F'_{xy}}{\partial F_{Xy}} & \dfrac{\partial  F'_{xy}}{\partial F_{xY}} & \dfrac{\partial  F'_{xy}}{\partial F_{xy}} & \dfrac{\partial  F'_{xy}}{\partial F_{XX}} & \dfrac{\partial  F'_{xy}}{\partial F_{Xx}} & \dfrac{\partial  F'_{xy}}{\partial F_{xx}} \\ 
		\dfrac{\partial  F'_{XX}}{\partial F_{XY}} & \dfrac{\partial  F'_{XX}}{\partial F_{Xy}} & \dfrac{\partial  F'_{XX}}{\partial F_{xY}} & \dfrac{\partial  F'_{XX}}{\partial F_{xy}} & \dfrac{\partial  F'_{XX}}{\partial F_{XX}} & \dfrac{\partial  F'_{XX}}{\partial F_{Xx}} & \dfrac{\partial  F'_{XX}}{\partial F_{xx}} \\ 
		\dfrac{\partial  F'_{Xx}}{\partial F_{XY}} & \dfrac{\partial  F'_{Xx}}{\partial F_{Xy}} & \dfrac{\partial  F'_{Xx}}{\partial F_{xY}} & \dfrac{\partial  F'_{Xx}}{\partial F_{xy}} & \dfrac{\partial  F'_{Xx}}{\partial F_{XX}} & \dfrac{\partial  F'_{Xx}}{\partial F_{Xx}} & \dfrac{\partial  F'_{Xx}}{\partial F_{xx}} \\ 
		\dfrac{\partial  F'_{xx}}{\partial F_{XY}} & \dfrac{\partial  F'_{xx}}{\partial F_{Xy}} & \dfrac{\partial  F'_{xx}}{\partial F_{xY}} & \dfrac{\partial  F'_{xx}}{\partial F_{xy}} & \dfrac{\partial  F'_{xx}}{\partial F_{XX}} & \dfrac{\partial  F'_{xx}}{\partial F_{Xx}} & \dfrac{\partial  F'_{xx}}{\partial F_{xx}} \\ 
	\end{array} \right)
\end{equation}


\noindent Analyzing the leading eigenvalue of $\mathbb{J}$ evaluated at the boundary equilibrium corresponding to initiation of a coevolutionary cycle (i.e., $q_y = 0$, $q_x = 1$) recovers the same invasion criteria as in the autosomal model ($\delta > 0$).

To determine when selection will favour invasion of the compensatory $x$ allele, we evaluated the candidate leading eigenvalue associated with corresponding invasion at $\mathbf{X}$, $\lambda_{\mathbf{X}}$ at the initial equilibrium where $q_x = 0$, and the $q_y$ is unspecified. Given these initial frequencies, and the fitness expressions provided in Table 1, $\lambda_{\mathbf{X}|\hat{q}_x = 0}$ is a large polynomial expression (see the accompanying Mathematica notebook in Appendix \hl{X}, where this result is derived). However, under additive fitness effects at $\mathbf{X}$ ($h_c = h_o = 1/2$), we can still solve $\lambda_{\mathbf{X}|\hat{q}_x = 0} - 1 = 0$ for $q_y$ to give the threshold frequency of the mutant $y$ chromosome at which selection will begin to favour the mutant $x$ compensatory allele. Although the solution is still complicated, it follows the general form

\begin{equation} \label{eq:xInvDelta-threshold}
	\tilde{q}_y^{\text{X}} = \frac{b - \sqrt{b^2 - 8 s_c c}} {2 c}
\end{equation}

\noindent (see Eq(\hl{6}) in the main text), where 

\begin{subequations}
	\begin{align}
		b &= s_m (2 + s_m (2 - s_m(1 - s_m))) - \delta (2 + s_m + s_m^2 + 2 s_m^3) + \delta^2(1 + s_m)^2 + s_c (2 + s_m^2 - \delta(1 + s_m)) \\
		c &= (s_c + s_m + 3 s_m^2 - 3 \delta(1 + s_m)) (s_m^2 - \delta(1 + s_m)).
	\end{align}
\end{subequations}

\noindent 

\newpage
%%%%%%%%%%%%%%%%%%%%%%%%%%%%%%%%%%%%%%%%%%%%%%%%
\section{Appendix B: Supplementary Results} \label{sec:SuppResults}
\renewcommand{\theequation}{B\arabic{equation}}
\setcounter{equation}{0}
\renewcommand{\thefigure}{B\arabic{figure}}
\setcounter{figure}{0}

%\begin{figure}[ht!]
%\centering
%\includegraphics[scale=0.58]{./FigB1-Gyno-obOut-funnel}
%\caption{Invasion of dominant male-sterility mutations into populations with segregating SA variation under obligate outcrossing. Plots show the fraction of parameter conditions maintaining single-locus SA polymorphism (within the range $0 < s_f,s_m \leq 0.5$) where a dominant male-sterility allele at $\mathbf{M}$ can invade populations initially at single-locus equilibrium frequencies for $\mathbf{A}$ with additive fitness effects ($h_f=h_m=1/2$). Results were obtained by evaluating the two relevant candidate leading eigenvalues ($\lambda_{\mathbf{M}}$,$\lambda_{\mathbf{AM}}$) of the Jacobian matrix of the genotype $\times$ transmission mode recursions for populations at the above initial conditions for $1000$ points uniformly distributed throughout the relevant $s_f \times s_m$ paramter space. Blue points indicate parameter sets where $1 - \lambda_{\mathbf{M}} > 0$, and/or $\lambda_{\mathbf{AM}} - 1 > 0$. Solid black lines represent the corresponding single-locus invasion criteria for SA alleles.}
%\label{fig:GynObOutFunnel}
%\end{figure}
%\newpage{}

%\begin{figure}[ht!]
%\centering
%\includegraphics[scale=0.58]{./FigB2-Gyno-C25-d80-funnel}
%\caption{Invasion of dominant male-sterility mutations into populations with segregating SA variation under conditions of low selfing ($C = 0.25$) and high inbreeding depression ($\delta = 0.8$). Plots show the fraction of parameter conditions maintaining single-locus SA polymorphism (within the range $0 < s_f,s_m \leq 0.5$) where a dominant male-sterility allele at $\mathbf{M}$ can invade populations initially at single-locus equilibrium frequencies for $\mathbf{A}$ with additive fitness effects ($h_f=h_m=1/2$). Results were obtained by evaluating the two relevant candidate leading eigenvalues ($\lambda_{\mathbf{M}}$,$\lambda_{\mathbf{AM}}$) of the Jacobian matrix of the genotype $\times$ transmission mode recursions for populations at the above initial conditions for $1000$ points uniformly distributed throughout the relevant $s_f \times s_m$ paramter space. Blue points indicate parameter sets where $1 - \lambda_{\mathbf{M}} > 0$, and/or $\lambda_{\mathbf{AM}} - 1 > 0$. Solid black lines represent the corresponding single-locus invasion criteria for SA alleles.}
%\label{fig:GynC25d80Funnel}
%\end{figure}
%\newpage{}
%\newpage{}

%\FloatBarrier


\clearpage

%%%%%%%%%%%%%%%%%%%%%
% Bibliography
%%%%%%%%%%%%%%%%%%%%%
%\bibliography{sexChromCoAdapt-Supplement-bibliography}



\end{document}
