%%%%%%%%%%%%%%%%%%%%%%%%%%%%%
%Preamble
\documentclass{article}

%Dependencies
\usepackage[left]{lineno}
\usepackage{titlesec}
\usepackage{color,soul}
\usepackage{ogonek}
\usepackage{float}


% Other Packages
%\usepackage{times}
\RequirePackage{fullpage}
\linespread{1.5}
\RequirePackage[colorlinks=true, allcolors=blue]{hyperref}
\RequirePackage[english]{babel}
\RequirePackage{amsmath,amsfonts,amssymb}
\RequirePackage[sc]{mathpazo}
\RequirePackage[T1]{fontenc}
\RequirePackage{url}

% Bibliography
%\usepackage[authoryear,sectionbib,sort]{natbib}
\usepackage{natbib} \bibpunct{(}{)}{;}{author-year}{}{,}
\bibliographystyle{amnatnat}
\addto{\captionsenglish}{\renewcommand{\refname}{Literature Cited}}
\setlength{\bibsep}{0.0pt}

% Graphics package
\usepackage{graphicx}
\graphicspath{{../output/figures/}.pdf}

% New commands: fonts
%\newcommand{\code}{\fontfamily{pcr}\selectfont}
%\newcommand*\chem[1]{\ensuremath{\mathrm{#1}}}
\newcommand\numberthis{\addtocounter{equation}{1}\tag{\theequation}}
\titleformat{\subsubsection}[runin]{\bfseries\itshape}{\thesubsubsection.}{0.5em}{}


%%%%%%%%%%%%%%%%%%%%%%%%%%%%%
% Title Page

\title{Sexually antagonistic coevolution between the sex chromosomes}
\author{Colin Olito$^{\ast}$, Jessica K.~Abbott, \& Katrine K.~Lund-Hansen}
\date{\today}

\begin{document}
\maketitle

\noindent{} $^{2}$ Department of Biology, Section for Evolutionary Ecology, Lund University, Lund 223 62, Sweden.

\noindent{} $^{\ast}$ Corresponding author e-mail: \url{colin.olito@gmail.com}

\bigskip

\noindent{} \textit{Manuscript elements}: Figure~1, Figure~2, Figure~3, Table~1; Online Supplementary Material: Appendix A -- Development of the recursions; Appendix B -- \hl{___}; Appendix C -- \hl{___}.

\bigskip
\noindent{} \textit{Running Head}: Coadaptation between the sex chromosomes

\bigskip

\noindent{} \textit{Keywords}: Coevolution; Compensatory evolution; Intralocus sexual conflict;  Sexual dimorphism; Sex chromosome evolution; Sperm Competition.

\bigskip

\noindent{} \textit{Manuscript type}: Major Article

\bigskip


% Set line number options
\linenumbers
\modulolinenumbers[1]
\renewcommand\linenumberfont{\normalfont\small}

%%%%%%%%%%%%%%%%%%%%%%%%%%%%%
% Main Text

\newpage{}
\section*{Abstract}

\noindent{} Blah blah blah.
\newpage{}


%%%%%%%%%%%%%%%%%%%%%%%%
\section*{Introduction}
%%%%%%%%%%%%%%%%%%%%%%%%


%%%%%%%%%%%%%%%%%%%%%%%%
\section*{Models} \label{sec:Models}
%%%%%%%%%%%%%%%%%%%%%%%%



%%%%%%%%%%%%%%%%%%%%%%%%
\subsection*{Subsection}


\begin{linenomath}\begin{align*} \label{eq:FprGyn}
    q^{f}_{auto}' &= ...  \\
    q^{m}_{auto}' &= ...  \\
    q_{y}' &= ..., \numberthis
\end{align*}\end{linenomath}

\noindent blah blah blah:

\begin{linenomath}\begin{align*} \label{eq:FprGyn}
    q^{f}_{x}' &= ...  \\
    q^{m}_{x}' &= ...  \\
    q_{y}' &= ..., \numberthis
\end{align*}\end{linenomath}


%%%%%%%%%%%%%%%%%%%%%%%%
\subsection*{Analyses} \label{subsec:analyses}

Blah blah blah...


\subsection*{Data availability}
A full development of all models can be found in Appendix A of the Online Supporting Information. Code necessary to reproduce the simulations is available at \textit{github link omitted to maintain author anonymity during peer review and will be provided upon acceptance}.%\url{https://github.com/colin-olito/sexChromCoAdapt}.

%%%%%%%%%%%%%%%%%%%%%%%%
\section*{Results}
%%%%%%%%%%%%%%%%%%%%%%%%

%%%%%%%%%%%%%%%%%%%%%%%%
\subsection*{subsection}

Blah blah blah...

\begin{equation}\label{eq:goodStuff}
	\lambda = the~good~stuff^2.
\end{equation}

\noindent Blah Blah BLah...


%%%%%%%%%%%%%%%%%%%%%%%%
\section*{Discussion}
%%%%%%%%%%%%%%%%%%%%%%%%

What it all means....

%%%%%%%%%%%%%%%%%%%%%%%%
\subsection*{Subsection}

Blah blah blah...


%%%%%%%%%%%%%%%%%%%%%%%%
\subsection*{Acknowledgements}
To be included upon acceptance...
%This research was supported by a Wenner-Gren postdoctoral Fellowship to CO, a Carl-Tryggers postdoctoral fellowship to KKL-H, and ERC-StG-2015-678148 and VR-2015-04680 to JKA. The authors thank C.~Venables, ..., for valuable discussion and feedback, as well as EDITOR, and XXX anonymous reviewers. JKA conceived the ideas explored in this study, and developed them with KKL-H and CO. CO developed the theoretical models and performed the analyses. All authors contributed to writing the manuscript.


%%%%%%%%%%%%%%%%%%%%%
% Bibliography
%%%%%%%%%%%%%%%%%%%%%
\bibliography{sexChromCoAdapt-bibliography}

\newpage


%%%%%%%%%%%%%%%%%%%%%%%%%%%%%%%%%%%%%%%%%%%%%%%%%%%%%%%%%%%%%%%%%%
%  Tables 

\begin{table}[htbp]
\centering
\caption{\bf Fitness expressions...}
\begin{tabular}{l c c c c} \hline
Haplotype & $ AM_1$ & $ AM_2$ & $ aM_1$ & $ aM_2$ \\
\hline
Female sex-function & & & & \\
$ AM_1$ & $1$ & $(1 + k)$ & $(1 - h_f s_f)$        & $(1 - h_f s_f)(1 + k)$ \\
$ AM_2$ & $-$ & $(1 + k)$ & $(1 - h_f s_f)(1 + k)$ & $(1 - h_f s_f)(1 + k)$ \\
$ aM_1$ & $-$ & $-$       & $(1 - s_f)$            & $(1 - s_f)(1 + k)$ \\
$ aM_2$ & $-$ & $-$       & $-$                    & $(1 - s_f)(1 + k)$ \\
Male sex-function & & & & \\
$ AM_1$ & $(1 - s_m)$ & $0$ & $(1 - h_m s_m)$ & $0$ \\
$ AM_2$ & $-$         & $0$ & $0$             & $0$ \\
$ aM_1$ & $-$         & $-$ & $1$             & $0$ \\
$ aM_2$ & $-$         & $-$ & $-$             & $0$ \\
\hline
\end{tabular}
\label{tab:fitness}\\
{\footnotesize Note: Rows and columns indicate the \textit{i}th and \textit{j}th gametic haplotype respectively. The lower triangle of each matrix is the reflection of the upper triangle, and is omitted for simplicity and consistency with the $i \geq j$ row/column indexing used throughout the article.}
\end{table}
\newpage{}



%%%%%%%%%%%%%%%%%%%%%%%%%%%%%%%%%%%%%%%%%%%%%%%%%%%%%%%%%%%%%%%%%%
%  Figures 

%\begin{figure}[htbp]
%\centering
%\includegraphics[scale=0.57]{./Fig1}
%\caption{Invasion of unisexuals into populations with pre-existing SA polymorphism. Plots show the fraction of parameter conditions maintaining single-locus SA polymorphism (within the range $0 < s_f,s_m \leq 0.5$) where a dominant sex-specific sterility allele at $\mathbf{M}$ can invade, assuming additive SA fitness effects at the SA $\mathbf{A}$ locus ($h_f=h_m=1/2$), plotted as a function of the recombination rate $r$. Panels A--C show results from the model of gynodioecy via invasion of a male-sterility allele, while planels D--F show results for the model of androdioecy via invasion of a female-sterility allele. For each panel, results are shown for different values of reproductive compensation, $k$, chosen as a fraction of the single-locus invasion threshold for the unisequal sterility allele ($\hat{k}$, which is equal to the right-hand side of Eq(\ref{eq:1LocGyn}) or Eq(\ref{eq:1LocAndro}) for the models of gynodioecy and androdioecy respectively). Hence, the orange, green, and dark blue lines show scenarios where unisexuals experience a decrease in gamete production relative to hermaphrodites of $1$, $5$, and $10\%$. Note the different scale for the x-axis in panels C and F. Results were obtained by evaluating the three candidate leading eigenvalues ($\lambda_{\mathbf{A}}$,$\lambda_{\mathbf{M}}$,$\lambda_{\mathbf{AM}}$) of the Jacobian matrix of the genotype $\times$ transmission mode recursions for populations at the above initial conditions for $1000$ points uniformly distributed throughout the relevant $s_f \times s_m$ parameter space.}
%\label{fig:PrInv}
%\end{figure}

%\newpage{}



\end{document}
